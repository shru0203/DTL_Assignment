\documentclass[12pt]{article}
\usepackage{amsmath}
\usepackage{graphicx}
\usepackage{booktabs} % For \toprule, \midrule and \bottomrule
\usepackage{siunitx} % Formats the units and values
\usepackage{pgfplotstable} % Generates table from .csv
% Setup siunitx:
\sisetup{
	round-mode          = places, % Rounds numbers
	round-precision     = 2, % to 2 places
}

\begin{document}
	\pagenumbering{gobble}
	\title{Maths Mid Term}
	\author{Shruti Ravichandran}
	\maketitle
	\newpage
	\tableofcontents
	\newpage
	{ \begin{center}
			College of Engineering Pune \\Department of Mathematics \\MA-19002 : Univariate Calculus \\End Semester Examination
		\end{center}
	}\raggedleft
	{\bf Date : 8th November 2022      \hspace*{\fill}            Max Marks : 60}
	\\{\bf Branches : All        \hspace*{\fill}                    Semester : 2}	       
	\\\raggedright{\bf Programme : BTech}   
	
	\pagenumbering{arabic}
	\addcontentsline{toc}{section}{Section A}
	\raggedright\section*{Section A}
	Q1.Solve the equation to obtain roots:
	\begin{equation*}
		x^2+2x+9=0
	\end{equation*}
	\\
	Q2.Solve the following inequation and write down the solution set:
	\begin{equation*}
		11x-4<15x+4<=13x+14,x\in W
	\end{equation*}
	\\
	Q3.State whether the following differential equations are linear or non linear,justify and solve.
	\begin{equation*}
		a)xy'+2y=e^{3x}/x,x>0 
		b)x^2ydy/dx-xy^2=1 
	\end{equation*}
	\\
	\addcontentsline{toc}{section}{Section B}
	\section*{Section B}
	Q4.Which of the given values of x and y will make the following pairs of matrices equal?
	\begin{equation*}
		\begin{bmatrix}3x+7&5\\y+1&2-3x
		\end{bmatrix}=
		\begin{bmatrix}0&y-2\\8&4
		\end{bmatrix}
	\end{equation*}
	\vspace{0.5cm}
	$\hspace{2cm}a)x=\frac{-1}{3},y=7 \hspace{5cm} b)Not possible\newline \hspace*{1.85cm}$
	$ c)y=7,x=\frac{-2}{3} \hspace{5cm} d)x=\frac{-1}{3},y=\frac{-2}{3}$
	\\
	Q5.Solve the following integral
	\begin{equation*}
		\int_{7}^8 (3x^4+2x+3x^2)dx
	\end{equation*}
	\\
	Q6. Evaluate the limit of the following
	\begin{equation*}
		\lim_{x \to +\infty}\sqrt[3]{x}+12x-2x^2
	\end{equation*}
	Q7.Solve the determinant
	\begin{equation*}
		\begin{vmatrix}3&4&5&9\\1&4&8&9\\2&6&84&6\\7&4&0&2
		\end{vmatrix}	
	\end{equation*}
	\addcontentsline{toc}{section}{Section C}
	\section*{Section C}
	
	Q8.The distribution in the table below shows the number of wickets taken by bowlers in one-day cricket matches. Find the mean number of wickets using the correct method. What does the mean signify?
	\begin{table}[h!]
		\begin{center}
			\caption{}
			\begin{tabular}{l|c} 
				\textbf{Number of wickets} & \textbf{Number of bowlers}\\
				\hline
				20-60 & 7\\
				60-100 &5\\
				100-150 & 16\\
				150-250 & 12\\
				250-300&2\\
				
				
			\end{tabular}
		\end{center}
	\end{table}\\
	\newpage
	Q9.Solve the following puzzle
	\vspace{1cm}
	\begin{center}
		
		\includegraphics{puzzle.jpg}\\
		
		\label{puzzle}
	\end{center}
	\vspace{1.5cm}
	Q10.Find the median height.
	\begin{table}[h!]
		\begin{center}
			\caption{Autogenerated table from .csv file.}
			\label{table1}
			\pgfplotstabletypeset[
			multicolumn names, % allows to have multicolumn names
			col sep=semicolon, % the seperator in our .csv file
			display columns/0/.style={
				column name=$Height$, % name of first column
				column type={S},string type},  % use siunitx for formatting
			display columns/1/.style={
				column name=$Number of girls$,
				column type={S},string type},
			every head row/.style={
				before row={\toprule}, % have a rule at top
				after row={
					\cm & \count\\ % the units seperated by &
					\midrule} % rule under units
			},
			every last row/.style={after row=\bottomrule}, % rule at bottom
\addtolength{\leftmargin}{0.2in} % sets up alignment with the following line.
\setlength{\itemindent}{-0.2in}
\bibitem[Bon96]{Boney96} Boney, L., Tewfik, A.H., and Hamdy, K.N., ``Digital
Watermarks for Audio Signals," \emph{Proceedings of the Third IEEE
	International Conference on Multimedia}, pp. 473-480, June 1996.
\bibitem[Goo94]{MG} Goossens, M., Mittelbach, F., Samarin, \emph{A LaTeX
	Companion}, Addison-Wesley, Reading, MA, 1994.
\bibitem[Kop99]{HK} Kopka, H., Daly P.W., \emph{A Guide to LaTeX},
Addison-Wesley, Reading, MA, 1999.
\bibitem[Pan98]{Pan} Pan, D., ``A Tutorial on MPEG/Audio Compression,"
\emph{IEEE Multimedia}, Vol.2, pp.60-74, Summer 1998.	]{table.csv} % filename/path to file
		\end{center}
	\end{table}
	\newpage
	\section{Differential Equations}
	\subsection{Types}
	\subsubsection{Ordinary Differential Equations.}
	\subsubsection{Partial Differential Equations.}
	\subsubsection{Linear Differential Equations.}
	\subsubsection{Non Linear Differential Equations.}
	\subsubsection{Homogenous Differential Equations.}
	\subsubsection{Non-Homogenous Differential Equations.}
	\newpage
\begin{thebibliography}{10}

\bibitem[Bon96]{Boney96} Boney, L., Tewfik, A.H., and Hamdy, K.N., ``Digital
Watermarks for Audio Signals," \emph{Proceedings of the Third IEEE
	International Conference on Multimedia}, pp. 473-480, June 1996.
\bibitem[Goo94]{MG} Goossens, M., Mittelbach, F., Samarin, \emph{A LaTeX
	Companion}, Addison-Wesley, Reading, MA, 1994.
\bibitem[Kop99]{HK} Kopka, H., Daly P.W., \emph{A Guide to LaTeX},
Addison-Wesley, Reading, MA, 1999.
\bibitem[Pan98]{Pan} Pan, D., ``A Tutorial on MPEG/Audio Compression,"
\emph{IEEE Multimedia}, Vol.2, pp.60-74, Summer 1998.
\end{thebibliography}
		
	
\end{document}